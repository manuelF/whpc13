\section{Ejercicio 1}

El ejercio 1 consiste en realizar un producto vectorial sencillo, empleando distintos m\'etodos.
Se utiliz\'o la libreria thrust para las versiones en C++ de CPU y de GPU, y se uso PyCUDA para
la version en Python.

La soluci\'on para la libreria thrust es utilizar la funci\'on transform, que toma varios parametros
de entrada, donde se definen los operandos, el lugar de destino y el operador. En nuestro caso,
se aplic\'a transform de la siguiente manera:

\texttt{
    thrust::transform(D1.begin(), D1.end(), D2.begin(), D3.begin(),
                       thrust::multiplies<float>());
}

Esto se define como: Tomar los datos de D1, de comienzo a fin, combinarlo con el dato de D2 al aplicarle la funci\'on 
de \texttt{thrust::multiplies<float>()} y dejarlo en D3. Una ventaja de thrust es que es totalmente 
portable entre CPU y GPU. Otra ventaja es que es muy idiomatico de C++, no hay ninguna construcci\'on muy rara
que no se encastre en el lenguaje. 


